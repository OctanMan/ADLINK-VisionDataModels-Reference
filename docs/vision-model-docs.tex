\documentclass{report}
\usepackage{array}
\usepackage{float}
\usepackage{tabularx}
\usepackage{longtable}
\usepackage{booktabs}
\usepackage[a4paper,margin=1cm]{geometry}
\usepackage[section]{placeins}
\usepackage{underscore}
%\usepackage{caption}
%\captionsetup{justification = raggedright, singlelinecheck=false}
\begin{document}
%\begin{landscape}
	\chapter{Introduction}
	This report contains a reference for the current datamodel (as represented by DataRiver TagGroups).
	\tableofcontents
	\listoftables
	\chapter{TagGroups -- Datamodel}
	In this chapter you will find the automatically generated documentation for the TagGroups. It is grouped by directory first, then per filename. This relies on the fact that the convention for TagGroup is to store them in a \textit{definitions} folder, which then in turn is further partitioned per \textit{context} and \textit{version}. This means that the result in this document will be shown in chronological order. 

	It is possible to define multiple TagGroups per file. This feature is primarily intented to allow for avoiding repetition (and therefore potential errors) when a certain datamodelling has a different QoS. However, this feature can be abused/misused and therefore lead to definitions of multiple independent TagGroups in one file. It also has the potential for multiple instances of a definition for a TagGroup, which is a bit of a maintenance nightmare. 

	To aid in detecting multiple definitions, this reference also includes a table of what TagGroup is defined in what file. Ideally, each TagGroup will only show up once in that table.
\section{../definitions/TagGroup/com.adlinktech.ai/v1.0}
\subsection{AITagGroup.json}

\begin{table}[H]
\begin{tabularx}{\textwidth}{l X} 
       \textbf{Name:} & AITagGroup \\ 
	   \textbf{Context:} & com.adlinktech.ai \\ 
	   \textbf{Version:} & v1.0 \\ 
	   \textbf{Description:} & no description available \\ 
	   \textbf{QoS:} & telemetry \\
	   \textbf{Toplevel Typename:} & \textit{No explicit name given} \\ 
\end{tabularx}
\caption{AITagGroup:com.adlinktech.ai:v1.0}\label{AITagGroup.json:table:AITagGroup}
\bigskip
\begin{tabularx}{\textwidth}{l l l l X} 
	 \textbf{Name} & \textbf{Unit} & \textbf{Type} & \textbf{Subtype} & \textbf{Description} \\
	 \midrule
   MachineID & None & UINT16 &  & Machine Identification \\
   ServiceID & None & UINT16 &  & Service Identification \\
   SourceID & None & UINT16 &  & Data source Identification \\
   AIName & None & STRING &  & AI Inference name ex: DEX100, Demo1 \\
   AICategory & None & STRING &  & AI Inference type ex: Classification, Detection, OCR \\
   Frame\_No & None & UINT64 &  & Frame Number \\
   ReceivedTimestamp & ms & UINT64 &  & UTC data received timestamp \\
   InferenceTime & ms & UINT64 &  & Inference duration \\
   SourceTimestamp & ms & UINT64 &  & UTC timestamp from source \\
   Resultformat & None & STRING &  & Name of the result type need to read. ex:Box , StringBox  \\
   StringResult & None & STRING\_SEQ &  & AI ouput array of string \\
   BoxResult & None & INT32\_SEQ &  & AI ouput array of (x,y) for two corner of box. ex: {0,0,120,240} is box with two corner (0,0) and  (120,240) \\
\end{tabularx}
\caption{AITagGroup:com.adlinktech.ai:v1.0: \textit{Toplevel Type (no explicit name given)}}\label{AITagGroup.json:table:AITagGroup-no-type-given}


\end{table}

\section{../definitions/TagGroup/com.adlinktech.usb2405/v1.0}
\subsection{ChannelDataTagGroup.json}

\begin{table}[H]
\begin{tabularx}{\textwidth}{l X} 
       \textbf{Name:} & raw\_data \\ 
	   \textbf{Context:} & com.adlinktech.usb2405 \\ 
	   \textbf{Version:} & v1.0 \\ 
	   \textbf{Description:} & MCM100 vibration data from channel \\ 
	   \textbf{QoS:} & vibration \\
	   \textbf{Toplevel Typename:} & \textit{No explicit name given} \\ 
\end{tabularx}
\caption{raw\_data:com.adlinktech.usb2405:v1.0}\label{ChannelDataTagGroup.json:table:raw-underscore-data}
\bigskip
\begin{tabularx}{\textwidth}{l l l l X} 
	 \textbf{Name} & \textbf{Unit} & \textbf{Type} & \textbf{Subtype} & \textbf{Description} \\
	 \midrule
   voltage & V & FLOAT32\_SEQ &  & Raw Voltage from USB 2405 DAQ \\
\end{tabularx}
\caption{raw\_data:com.adlinktech.usb2405:v1.0: \textit{Toplevel Type (no explicit name given)}}\label{ChannelDataTagGroup.json:table:raw-underscore-data-no-type-given}


\end{table}

\subsection{ChannelMetaDataTagGroup.json}

\begin{table}[H]
\begin{tabularx}{\textwidth}{l X} 
       \textbf{Name:} & meta\_data \\ 
	   \textbf{Context:} & com.adlinktech.usb2405 \\ 
	   \textbf{Version:} & v1.0 \\ 
	   \textbf{Description:} & MCM100 meta data from channel \\ 
	   \textbf{QoS:} & state \\
	   \textbf{Toplevel Typename:} & \textit{No explicit name given} \\ 
\end{tabularx}
\caption{meta\_data:com.adlinktech.usb2405:v1.0}\label{ChannelMetaDataTagGroup.json:table:meta-underscore-data}
\bigskip
\begin{tabularx}{\textwidth}{l l l l X} 
	 \textbf{Name} & \textbf{Unit} & \textbf{Type} & \textbf{Subtype} & \textbf{Description} \\
	 \midrule
   SampleRate & N/A & INT64 &  & The sample rate that DAQ channel use \\
   DataCount & N/A & INT64 &  & The data length that DAQ acquire once \\
\end{tabularx}
\caption{meta\_data:com.adlinktech.usb2405:v1.0: \textit{Toplevel Type (no explicit name given)}}\label{ChannelMetaDataTagGroup.json:table:meta-underscore-data-no-type-given}


\end{table}

\subsection{DAQStateTagGroup.json}

\begin{table}[H]
\begin{tabularx}{\textwidth}{l X} 
       \textbf{Name:} & configuration \\ 
	   \textbf{Context:} & com.adlinktech.usb2405 \\ 
	   \textbf{Version:} & v1.0 \\ 
	   \textbf{Description:} & USB2405 data acquisition information \\ 
	   \textbf{QoS:} & state \\
	   \textbf{Toplevel Typename:} & \textit{No explicit name given} \\ 
\end{tabularx}
\caption{configuration:com.adlinktech.usb2405:v1.0}\label{DAQStateTagGroup.json:table:configuration}
\bigskip
\begin{tabularx}{\textwidth}{l l l l X} 
	 \textbf{Name} & \textbf{Unit} & \textbf{Type} & \textbf{Subtype} & \textbf{Description} \\
	 \midrule
   ChannelNo & N/A & INT16 &  & channel active status \\
   active & N/A & BOOLEAN &  & channel active status \\
   flowId & N/A & STRING &  & the flowID channel use \\
   EnableIEPE & N/A & BOOLEAN &  & channel IEPE status \\
   InputType & N/A & STRING &  & channel input type \\
   CoupleType & N/A & STRING &  & channel Couple type \\
\end{tabularx}
\caption{configuration:com.adlinktech.usb2405:v1.0: ChannelStatus}\label{DAQStateTagGroup.json:table:configuration-ChannelStatus}

\bigskip
\begin{tabularx}{\textwidth}{l l l l X} 
	 \textbf{Name} & \textbf{Unit} & \textbf{Type} & \textbf{Subtype} & \textbf{Description} \\
	 \midrule
   ConversionSource & N/A & STRING &  & channel active status \\
   Mode & N/A & STRING &  & channel active status \\
   Source & N/A & STRING &  & the flowID channel use \\
   Polarity & N/A & STRING &  & channel IEPE status \\
   DLY1Cnt & N/A & INT16 &  & channel input type \\
   DLY2Cnt & N/A & INT16 &  & channel input type \\
   Level & N/A & INT16 &  & channel input type \\
\end{tabularx}
\caption{configuration:com.adlinktech.usb2405:v1.0: TriggerModeSetting}\label{DAQStateTagGroup.json:table:configuration-TriggerModeSetting}

\bigskip
\begin{tabularx}{\textwidth}{l l l l X} 
	 \textbf{Name} & \textbf{Unit} & \textbf{Type} & \textbf{Subtype} & \textbf{Description} \\
	 \midrule
   CardID & N/A & INT16 &  & The card id that DAQ use \\
   SampleRate & N/A & INT64 &  & The sample rate that DAQ channel use \\
   DataCount & N/A & INT64 &  & The data length that DAQ acquire once \\
   ChannelInfo\_A0 & n/a & NVP\_SEQ & ChannelStatus & channel 0 configurations \\
   ChannelInfo\_A1 & n/a & NVP\_SEQ & ChannelStatus & channel 1 configurations \\
   ChannelInfo\_A2 & n/a & NVP\_SEQ & ChannelStatus & channel 2 configurations \\
   ChannelInfo\_A3 & n/a & NVP\_SEQ & ChannelStatus & channel 3 configurations \\
   TriggerMode & N/A & NVP\_SEQ & TriggerModeSetting & The trigger mode that DAQ use \\
\end{tabularx}
\caption{configuration:com.adlinktech.usb2405:v1.0: \textit{Toplevel Type (no explicit name given)}}\label{DAQStateTagGroup.json:table:configuration-no-type-given}


\end{table}

\section{../definitions/TagGroup/com.adlinktech.vision/0.1.0}
\subsection{CreateStreamTagGroup.json}

\begin{table}[H]
\begin{tabularx}{\textwidth}{l X} 
       \textbf{Name:} & CreateStream \\ 
	   \textbf{Context:} & com.adlinktech.vision \\ 
	   \textbf{Version:} & 0.1.0 \\ 
	   \textbf{Description:} & Request the creation of a stream with given configuration. Stream's are unique within a Stream Viewer per StreamId and Inference output type combination. Target the correct Stream Viewer by setting the FlowId of this DataSample to the ContextId of the Stream Viewer. \\ 
	   \textbf{QoS:} & state \\
	   \textbf{Toplevel Typename:} & \textit{No explicit name given} \\ 
\end{tabularx}
\caption{CreateStream:com.adlinktech.vision:0.1.0}\label{CreateStreamTagGroup.json:table:CreateStream}
\bigskip
\begin{tabularx}{\textwidth}{l l l l X} 
	 \textbf{Name} & \textbf{Unit} & \textbf{Type} & \textbf{Subtype} & \textbf{Description} \\
	 \midrule
   stream\_id & n/a & STRING &  & The streamId that the Stream Viewer should read Vision data from. \\
   inference\_output\_type & n/a & STRING &  & What kind of Inference Engine output type to apply to the stream. \\
\end{tabularx}
\caption{CreateStream:com.adlinktech.vision:0.1.0: \textit{Toplevel Type (no explicit name given)}}\label{CreateStreamTagGroup.json:table:CreateStream-no-type-given}


\end{table}

\subsection{DeleteStreamTagGroup.json}

\begin{table}[H]
\begin{tabularx}{\textwidth}{l X} 
       \textbf{Name:} & DeleteStream \\ 
	   \textbf{Context:} & com.adlinktech.vision \\ 
	   \textbf{Version:} & 0.1.0 \\ 
	   \textbf{Description:} & Request the deletion of a stream. Target the correct Stream Viewer by setting the FlowId of this DataSample to the ContextId of the Stream Viewer. \\ 
	   \textbf{QoS:} & state \\
	   \textbf{Toplevel Typename:} & \textit{No explicit name given} \\ 
\end{tabularx}
\caption{DeleteStream:com.adlinktech.vision:0.1.0}\label{DeleteStreamTagGroup.json:table:DeleteStream}
\bigskip
\begin{tabularx}{\textwidth}{l l l l X} 
	 \textbf{Name} & \textbf{Unit} & \textbf{Type} & \textbf{Subtype} & \textbf{Description} \\
	 \midrule
   stream\_id & n/a & STRING &  & The streamId of the stream that should be deleted. \\
   inference\_output\_type & n/a & STRING &  & The Inference Engine output type of the stream to be deleted. \\
\end{tabularx}
\caption{DeleteStream:com.adlinktech.vision:0.1.0: \textit{Toplevel Type (no explicit name given)}}\label{DeleteStreamTagGroup.json:table:DeleteStream-no-type-given}


\end{table}

\subsection{StreamViewerConfigTagGroup.json}

\begin{table}[H]
\begin{tabularx}{\textwidth}{l X} 
       \textbf{Name:} & StreamViewerConfig \\ 
	   \textbf{Context:} & com.adlinktech.vision \\ 
	   \textbf{Version:} & 0.1.0 \\ 
	   \textbf{Description:} & Current configuration of this Stream Viewer instance. \\ 
	   \textbf{QoS:} & state \\
	   \textbf{Toplevel Typename:} & \textit{No explicit name given} \\ 
\end{tabularx}
\caption{StreamViewerConfig:com.adlinktech.vision:0.1.0}\label{StreamViewerConfigTagGroup.json:table:StreamViewerConfig}
\bigskip
\begin{tabularx}{\textwidth}{l l l l X} 
	 \textbf{Name} & \textbf{Unit} & \textbf{Type} & \textbf{Subtype} & \textbf{Description} \\
	 \midrule
   stream\_id & n/a & STRING &  & The streamId of the ADLINK Vision stream frame and inference data is being read from. \\
   inference\_output\_type & n/a & STRING &  & What kind of Inference Engine output type is being applied to the stream. \\
   rtsp\_streaming\_address & n/a & STRING &  & The address this stream is streaming from. \\
\end{tabularx}
\caption{StreamViewerConfig:com.adlinktech.vision:0.1.0: StreamConfig}\label{StreamViewerConfigTagGroup.json:table:StreamViewerConfig-StreamConfig}

\bigskip
\begin{tabularx}{\textwidth}{l l l l X} 
	 \textbf{Name} & \textbf{Unit} & \textbf{Type} & \textbf{Subtype} & \textbf{Description} \\
	 \midrule
   streams & n/a & NVP\_SEQ & StreamConfig & A sequence of the streams this Stream Viewer is streaming. \\
\end{tabularx}
\caption{StreamViewerConfig:com.adlinktech.vision:0.1.0: \textit{Toplevel Type (no explicit name given)}}\label{StreamViewerConfigTagGroup.json:table:StreamViewerConfig-no-type-given}


\end{table}

\section{../definitions/TagGroup/com.adlinktech.vision/v1.0}
\subsection{AcknowledgeTagGroup.json}

\begin{table}[H]
\begin{tabularx}{\textwidth}{l X} 
       \textbf{Name:} & Acknowledge \\ 
	   \textbf{Context:} & com.adlinktech.vision \\ 
	   \textbf{Version:} & v1.0 \\ 
	   \textbf{Description:} & Confirmation of Alarm events \\ 
	   \textbf{QoS:} & event \\
	   \textbf{Toplevel Typename:} & \textit{No explicit name given} \\ 
\end{tabularx}
\caption{Acknowledge:com.adlinktech.vision:v1.0}\label{AcknowledgeTagGroup.json:table:Acknowledge}
\bigskip
\begin{tabularx}{\textwidth}{l l l l X} 
	 \textbf{Name} & \textbf{Unit} & \textbf{Type} & \textbf{Subtype} & \textbf{Description} \\
	 \midrule
   id & UUID & STRING &  & Alarm ID for alarm being Acknowledged \\
   stream\_id & UUID & STRING &  & Service ID publishing the Alarm ack. \\
   acknowledge & N/A & BOOLEAN &  & State of alarm acknowledgement (True: Valid Alarm. False: Not valid) \\
   message & n/a & STRING &  & Alarm information text \\
\end{tabularx}
\caption{Acknowledge:com.adlinktech.vision:v1.0: \textit{Toplevel Type (no explicit name given)}}\label{AcknowledgeTagGroup.json:table:Acknowledge-no-type-given}


\end{table}

\subsection{AlarmTagGroup.json}

\begin{table}[H]
\begin{tabularx}{\textwidth}{l X} 
       \textbf{Name:} & Alarm \\ 
	   \textbf{Context:} & com.adlinktech.vision \\ 
	   \textbf{Version:} & v1.0 \\ 
	   \textbf{Description:} & Alarms \& alert data stream \\ 
	   \textbf{QoS:} & event \\
	   \textbf{Toplevel Typename:} & \textit{No explicit name given} \\ 
\end{tabularx}
\caption{Alarm:com.adlinktech.vision:v1.0}\label{AlarmTagGroup.json:table:Alarm}
\bigskip
\begin{tabularx}{\textwidth}{l l l l X} 
	 \textbf{Name} & \textbf{Unit} & \textbf{Type} & \textbf{Subtype} & \textbf{Description} \\
	 \midrule
   id & UUID & STRING &  & Alarm ID \\
   stream\_id & UUID & STRING &  & Service ID publishing the Alarm \\
   target & UUID & STRING &  & Target FlowID of the HMI where alarm will be displayed \\
   level & com.vision.models.alarm.level & STRING &  & Alarm Level enum \\
   timestamp & timestamp & STRING &  & Alarm Event Timestamp \\
   message & n/a & STRING &  & Alarm information text \\
   meta & N/A & STRING &  & Serialized value \\
\end{tabularx}
\caption{Alarm:com.adlinktech.vision:v1.0: \textit{Toplevel Type (no explicit name given)}}\label{AlarmTagGroup.json:table:Alarm-no-type-given}


\end{table}

\subsection{BinaryArtifactTagGroup.json}

\begin{table}[H]
\begin{tabularx}{\textwidth}{l X} 
       \textbf{Name:} & BinaryArtifact \\ 
	   \textbf{Context:} & com.adlinktech.vision \\ 
	   \textbf{Version:} & v1.0 \\ 
	   \textbf{Description:} & Message containing serialized file artifact \\ 
	   \textbf{QoS:} & event \\
	   \textbf{Toplevel Typename:} & \textit{No explicit name given} \\ 
\end{tabularx}
\caption{BinaryArtifact:com.adlinktech.vision:v1.0}\label{BinaryArtifactTagGroup.json:table:BinaryArtifact}
\bigskip
\begin{tabularx}{\textwidth}{l l l l X} 
	 \textbf{Name} & \textbf{Unit} & \textbf{Type} & \textbf{Subtype} & \textbf{Description} \\
	 \midrule
   source & N/A & STRING &  & Artifact origin \\
   target & N/A & STRING &  & (Optional) URI/ID of the intended recepient/use case \\
   name & N/A & STRING &  & Artifact name \\
   size & bytes & UINT64 &  & Data size \\
   kind & N/A & STRING &  & Artifact description \\
   timestamp & time & UINT64 &  & Updated/Created timestamp \\
   data & n/a & BYTE\_SEQ &  & List of Detection Box Data (the results) \\
\end{tabularx}
\caption{BinaryArtifact:com.adlinktech.vision:v1.0: \textit{Toplevel Type (no explicit name given)}}\label{BinaryArtifactTagGroup.json:table:BinaryArtifact-no-type-given}


\end{table}

\subsection{CameraConnectRequestTagGroup.json}

\begin{table}[H]
\begin{tabularx}{\textwidth}{l X} 
       \textbf{Name:} & CameraConnectRequest \\ 
	   \textbf{Context:} & com.adlinktech.vision \\ 
	   \textbf{Version:} & v1.0 \\ 
	   \textbf{Description:} & Inference engine results for classification model \\ 
	   \textbf{QoS:} & event \\
	   \textbf{Toplevel Typename:} & \textit{No explicit name given} \\ 
\end{tabularx}
\caption{CameraConnectRequest:com.adlinktech.vision:v1.0}\label{CameraConnectRequestTagGroup.json:table:CameraConnectRequest}
\bigskip
\begin{tabularx}{\textwidth}{l l l l X} 
	 \textbf{Name} & \textbf{Unit} & \textbf{Type} & \textbf{Subtype} & \textbf{Description} \\
	 \midrule
   id & n/a & STRING &  & Unique camera id \\
   uri & n/a & STRING &  & Camera network URI \\
   name & n/a & STRING &  & Camera network URI \\
   action & n/a & STRING &  & ADD or REMOVE a network camera from a remote service \\
   status & n/a & STRING &  & Status string of performed action \\
   encoding & Enum & STRING &  & See com::vision::models::CompressionKind \\
   port & Value & UINT32 &  & Port number \\
\end{tabularx}
\caption{CameraConnectRequest:com.adlinktech.vision:v1.0: \textit{Toplevel Type (no explicit name given)}}\label{CameraConnectRequestTagGroup.json:table:CameraConnectRequest-no-type-given}


\end{table}

\subsection{CaptureStartTagGroup.json}

\begin{table}[H]
\begin{tabularx}{\textwidth}{l X} 
       \textbf{Name:} & CaptureStart \\ 
	   \textbf{Context:} & com.adlinktech.vision \\ 
	   \textbf{Version:} & v1.0 \\ 
	   \textbf{Description:} & Perform a capture request for start. \\ 
	   \textbf{QoS:} & event \\
	   \textbf{Toplevel Typename:} & \textit{No explicit name given} \\ 
\end{tabularx}
\caption{CaptureStart:com.adlinktech.vision:v1.0}\label{CaptureStartTagGroup.json:table:CaptureStart}
\bigskip
\begin{tabularx}{\textwidth}{l l l l X} 
	 \textbf{Name} & \textbf{Unit} & \textbf{Type} & \textbf{Subtype} & \textbf{Description} \\
	 \midrule
   StreamId & NA & STRING &  & The stream id would like to start \\
   Timeout & NA & INT32 &  & Stream capture timeout time, 0 for no timeout. \\
   NumberOfFrame & NA & INT32 &  & How manay frames per time unit, 0 means no downsampling \\
   TimeUnit & NA & STRING &  & Time unit, could be Second/Minute/Hour \\
\end{tabularx}
\caption{CaptureStart:com.adlinktech.vision:v1.0: \textit{Toplevel Type (no explicit name given)}}\label{CaptureStartTagGroup.json:table:CaptureStart-no-type-given}


\end{table}

\subsection{CaptureStopTagGroup.json}

\begin{table}[H]
\begin{tabularx}{\textwidth}{l X} 
       \textbf{Name:} & CaptureStop \\ 
	   \textbf{Context:} & com.adlinktech.vision \\ 
	   \textbf{Version:} & v1.0 \\ 
	   \textbf{Description:} & Perform a capture request for stop. \\ 
	   \textbf{QoS:} & event \\
	   \textbf{Toplevel Typename:} & \textit{No explicit name given} \\ 
\end{tabularx}
\caption{CaptureStop:com.adlinktech.vision:v1.0}\label{CaptureStopTagGroup.json:table:CaptureStop}
\bigskip
\begin{tabularx}{\textwidth}{l l l l X} 
	 \textbf{Name} & \textbf{Unit} & \textbf{Type} & \textbf{Subtype} & \textbf{Description} \\
	 \midrule
   StreamId & NA & STRING &  & The stream id would like to stop \\
\end{tabularx}
\caption{CaptureStop:com.adlinktech.vision:v1.0: \textit{Toplevel Type (no explicit name given)}}\label{CaptureStopTagGroup.json:table:CaptureStop-no-type-given}


\end{table}

\subsection{ClassificationTagGroup.json}

\begin{table}[H]
\begin{tabularx}{\textwidth}{l X} 
       \textbf{Name:} & Classification \\ 
	   \textbf{Context:} & com.adlinktech.vision \\ 
	   \textbf{Version:} & v1.0 \\ 
	   \textbf{Description:} & Inference engine results for classification model \\ 
	   \textbf{QoS:} & telemetry \\
	   \textbf{Toplevel Typename:} & \textit{No explicit name given} \\ 
\end{tabularx}
\caption{Classification:com.adlinktech.vision:v1.0}\label{ClassificationTagGroup.json:table:Classification}
\bigskip
\begin{tabularx}{\textwidth}{l l l l X} 
	 \textbf{Name} & \textbf{Unit} & \textbf{Type} & \textbf{Subtype} & \textbf{Description} \\
	 \midrule
   index & n/a & INT32 &  & Classification index \\
   output & n/a & STRING &  & Output type - used when classification model has multiple types of labels for each output index \\
   label & n/a & STRING &  & Classification label name \\
   probability & Percentage & FLOAT32 &  & Network confidence \\
\end{tabularx}
\caption{Classification:com.adlinktech.vision:v1.0: ClassificationData}\label{ClassificationTagGroup.json:table:Classification-ClassificationData}

\bigskip
\begin{tabularx}{\textwidth}{l l l l X} 
	 \textbf{Name} & \textbf{Unit} & \textbf{Type} & \textbf{Subtype} & \textbf{Description} \\
	 \midrule
   engine\_id & UUID & STRING &  & Inference engine identifier \\
   stream\_id & UUID & STRING &  & ID of the stream fed into the inference engine \\
   frame\_id & NUM & UINT32 &  & ID of the input video frame fed to the inference engine \\
   data & n/a & NVP\_SEQ & ClassificationData & List of Classification Data (the results) \\
\end{tabularx}
\caption{Classification:com.adlinktech.vision:v1.0: \textit{Toplevel Type (no explicit name given)}}\label{ClassificationTagGroup.json:table:Classification-no-type-given}


\end{table}

\subsection{ConfigActualTagGroup.json}

\begin{table}[H]
\begin{tabularx}{\textwidth}{l X} 
       \textbf{Name:} & ConfigActual \\ 
	   \textbf{Context:} & com.adlinktech.vision \\ 
	   \textbf{Version:} & v1.0 \\ 
	   \textbf{Description:} & Current values of configuration topic \\ 
	   \textbf{QoS:} & state \\
	   \textbf{Toplevel Typename:} & \textit{No explicit name given} \\ 
\end{tabularx}
\caption{ConfigActual:com.adlinktech.vision:v1.0}\label{ConfigActualTagGroup.json:table:ConfigActual}
\bigskip
\begin{tabularx}{\textwidth}{l l l l X} 
	 \textbf{Name} & \textbf{Unit} & \textbf{Type} & \textbf{Subtype} & \textbf{Description} \\
	 \midrule
   type & n/a & STRING &  & Config Request Type \\
   payload & n/a & STRING &  & Current config values \\
\end{tabularx}
\caption{ConfigActual:com.adlinktech.vision:v1.0: ConfigData}\label{ConfigActualTagGroup.json:table:ConfigActual-ConfigData}

\bigskip
\begin{tabularx}{\textwidth}{l l l l X} 
	 \textbf{Name} & \textbf{Unit} & \textbf{Type} & \textbf{Subtype} & \textbf{Description} \\
	 \midrule
   source & UUID & STRING &  & Application ID effecting the config request \\
   data & n/a & NVP\_SEQ & ConfigData & List of Detection Box Data (the results) \\
\end{tabularx}
\caption{ConfigActual:com.adlinktech.vision:v1.0: \textit{Toplevel Type (no explicit name given)}}\label{ConfigActualTagGroup.json:table:ConfigActual-no-type-given}


\end{table}

\subsection{ConfigRequestTagGroup.json}

\begin{table}[H]
\begin{tabularx}{\textwidth}{l X} 
       \textbf{Name:} & ConfigRequest \\ 
	   \textbf{Context:} & com.adlinktech.vision \\ 
	   \textbf{Version:} & v1.0 \\ 
	   \textbf{Description:} & Requested configuration for config effector \\ 
	   \textbf{QoS:} & event \\
	   \textbf{Toplevel Typename:} & \textit{No explicit name given} \\ 
\end{tabularx}
\caption{ConfigRequest:com.adlinktech.vision:v1.0}\label{ConfigRequestTagGroup.json:table:ConfigRequest}
\bigskip
\begin{tabularx}{\textwidth}{l l l l X} 
	 \textbf{Name} & \textbf{Unit} & \textbf{Type} & \textbf{Subtype} & \textbf{Description} \\
	 \midrule
   id & UUID & STRING &  & Target host id to effect the config request \\
   request & n/a & UINT32 &  & Request ID \\
   type & n/a & STRING &  & Config Request Type \\
   payload & n/a & STRING &  & Request data \\
\end{tabularx}
\caption{ConfigRequest:com.adlinktech.vision:v1.0: \textit{Toplevel Type (no explicit name given)}}\label{ConfigRequestTagGroup.json:table:ConfigRequest-no-type-given}


\end{table}

\subsection{DetectionBoxTagGroup.json}

\begin{table}[H]
\begin{tabularx}{\textwidth}{l X} 
       \textbf{Name:} & DetectionBox \\ 
	   \textbf{Context:} & com.adlinktech.vision \\ 
	   \textbf{Version:} & v1.0 \\ 
	   \textbf{Description:} & Inference engine results for object detection model outputing bounding boxes \\ 
	   \textbf{QoS:} & telemetry \\
	   \textbf{Toplevel Typename:} & \textit{No explicit name given} \\ 
\end{tabularx}
\caption{DetectionBox:com.adlinktech.vision:v1.0}\label{DetectionBoxTagGroup.json:table:DetectionBox}
\bigskip
\begin{tabularx}{\textwidth}{l l l l X} 
	 \textbf{Name} & \textbf{Unit} & \textbf{Type} & \textbf{Subtype} & \textbf{Description} \\
	 \midrule
   obj\_id & UUID & INT32 &  & Detected object id \\
   obj\_label & UUID & STRING &  & Detected object proper name \\
   class\_id & UUID & INT32 &  & Detected object's classification type as raw id \\
   class\_label & UUID & STRING &  & Detected object's classification as proper name \\
   x1 & Percentage & FLOAT32 &  & Top Left X Coordinate (\% from 0,0) \\
   y1 & Percentage & FLOAT32 &  & Top Left Y Coordinate (\% from 0,0) \\
   x2 & Percentage & FLOAT32 &  & Bottom Right X Coordinate (\% from 0,0) \\
   y2 & Percentage & FLOAT32 &  & Bottom Right Y Coordinate (\% from 0,0) \\
   probability & Percentage & FLOAT32 &  & Network confidence \\
   meta & N/A & STRING &  & Buffer for extra inference metadata \\
\end{tabularx}
\caption{DetectionBox:com.adlinktech.vision:v1.0: DetectionBoxData}\label{DetectionBoxTagGroup.json:table:DetectionBox-DetectionBoxData}

\bigskip
\begin{tabularx}{\textwidth}{l l l l X} 
	 \textbf{Name} & \textbf{Unit} & \textbf{Type} & \textbf{Subtype} & \textbf{Description} \\
	 \midrule
   engine\_id & UUID & STRING &  & Inference engine identifier \\
   stream\_id & UUID & STRING &  & ID of the stream fed into the inference engine \\
   frame\_id & NUM & UINT32 &  & ID of the input video frame fed to the inference engine \\
   data & n/a & NVP\_SEQ & DetectionBoxData & List of Detection Box Data (the results) \\
\end{tabularx}
\caption{DetectionBox:com.adlinktech.vision:v1.0: \textit{Toplevel Type (no explicit name given)}}\label{DetectionBoxTagGroup.json:table:DetectionBox-no-type-given}


\end{table}

\subsection{DetectionPointTagGroup.json}

\begin{table}[H]
\begin{tabularx}{\textwidth}{l X} 
       \textbf{Name:} & DetectionPoint \\ 
	   \textbf{Context:} & com.adlinktech.vision \\ 
	   \textbf{Version:} & v1.0 \\ 
	   \textbf{Description:} & Inference engine results for object detection model outputing points \\ 
	   \textbf{QoS:} & telemetry \\
	   \textbf{Toplevel Typename:} & \textit{No explicit name given} \\ 
\end{tabularx}
\caption{DetectionPoint:com.adlinktech.vision:v1.0}\label{DetectionPointTagGroup.json:table:DetectionPoint}
\bigskip
\begin{tabularx}{\textwidth}{l l l l X} 
	 \textbf{Name} & \textbf{Unit} & \textbf{Type} & \textbf{Subtype} & \textbf{Description} \\
	 \midrule
   obj\_id & UUID & INT32 &  & Detected object id \\
   obj\_label & UUID & STRING &  & Detected object proper name \\
   class\_id & UUID & INT32 &  & Detected object's classification type as raw id \\
   class\_label & UUID & STRING &  & Detected object's classification as proper name \\
   x & Percentage & FLOAT32 &  & Center X Coordinate (\% from 0,0) \\
   y & Percentage & FLOAT32 &  & Center Y Coordinate (\% from 0,0) \\
   radius & Percentage & FLOAT32 &  & Size of the point (\% of image width) \\
   probability & Percentage & FLOAT32 &  & Network confidence \\
   meta & N/A & STRING &  & Buffer for extra inference metadata \\
\end{tabularx}
\caption{DetectionPoint:com.adlinktech.vision:v1.0: DetectionPointData}\label{DetectionPointTagGroup.json:table:DetectionPoint-DetectionPointData}

\bigskip
\begin{tabularx}{\textwidth}{l l l l X} 
	 \textbf{Name} & \textbf{Unit} & \textbf{Type} & \textbf{Subtype} & \textbf{Description} \\
	 \midrule
   engine\_id & UUID & STRING &  & Inference engine identifier \\
   stream\_id & UUID & STRING &  & ID of the stream fed into the inference engine \\
   frame\_id & NUM & UINT32 &  & ID of the input video frame fed to the inference engine \\
   data & n/a & NVP\_SEQ & DetectionPointData & List of Detection Point Data (the results) \\
\end{tabularx}
\caption{DetectionPoint:com.adlinktech.vision:v1.0: \textit{Toplevel Type (no explicit name given)}}\label{DetectionPointTagGroup.json:table:DetectionPoint-no-type-given}


\end{table}

\subsection{DeviceErrorTagGroup.json}

\begin{table}[H]
\begin{tabularx}{\textwidth}{l X} 
       \textbf{Name:} & DeviceError \\ 
	   \textbf{Context:} & com.adlinktech.vision \\ 
	   \textbf{Version:} & v1.0 \\ 
	   \textbf{Description:} & Device error log \\ 
	   \textbf{QoS:} & event \\
	   \textbf{Toplevel Typename:} & \textit{No explicit name given} \\ 
\end{tabularx}
\caption{DeviceError:com.adlinktech.vision:v1.0}\label{DeviceErrorTagGroup.json:table:DeviceError}
\bigskip
\begin{tabularx}{\textwidth}{l l l l X} 
	 \textbf{Name} & \textbf{Unit} & \textbf{Type} & \textbf{Subtype} & \textbf{Description} \\
	 \midrule
   device\_id & UUID & STRING &  & Publishing Device \\
   message & n/a & STRING &  & Concise error message \\
   error\_id & n/a & INT32 &  & Error identifier \\
   description & n/a & STRING &  & Error description \\
\end{tabularx}
\caption{DeviceError:com.adlinktech.vision:v1.0: \textit{Toplevel Type (no explicit name given)}}\label{DeviceErrorTagGroup.json:table:DeviceError-no-type-given}


\end{table}

\subsection{DeviceInfoTagGroup.json}

\begin{table}[H]
\begin{tabularx}{\textwidth}{l X} 
       \textbf{Name:} & DeviceInfo \\ 
	   \textbf{Context:} & com.adlinktech.vision \\ 
	   \textbf{Version:} & v1.0 \\ 
	   \textbf{Description:} & Vision Device properties \\ 
	   \textbf{QoS:} & state \\
	   \textbf{Toplevel Typename:} & \textit{No explicit name given} \\ 
\end{tabularx}
\caption{DeviceInfo:com.adlinktech.vision:v1.0}\label{DeviceInfoTagGroup.json:table:DeviceInfo}
\bigskip
\begin{tabularx}{\textwidth}{l l l l X} 
	 \textbf{Name} & \textbf{Unit} & \textbf{Type} & \textbf{Subtype} & \textbf{Description} \\
	 \midrule
   stream\_id & UUID & STRING &  & Stream publisher ID \\
   mac\_address & n/a & STRING &  & Host address \\
   ip\_address & n/a & STRING &  & Host machine IP Address \\
   port & n/a & INT32 &  & Connection port (optional) \\
   uri & n/a & STRING &  & Video Interface URI (rtsp://xx/h264) \\
   manufacturer & n/a & STRING &  & Vision Device manufacturer \\
   model & n/a & STRING &  & Vision Device model \\
   serial & n/a & STRING &  & Vision Device serial identifier \\
   fw\_version & n/a & STRING &  & Vision Device firmware version \\
   dev\_id & n/a & STRING &  & Vision Device host interface (ex. /dev/video0 or /dev/ttyUSB0) \\
   status & DeviceStatus & STRING &  & DeviceStatus enum \\
   kind & DeviceKind & STRING &  & Vision device kind enum \\
   protocol & ProtocolKind & STRING &  & ProtocolKind enum describing how the device communicates \\
\end{tabularx}
\caption{DeviceInfo:com.adlinktech.vision:v1.0: \textit{Toplevel Type (no explicit name given)}}\label{DeviceInfoTagGroup.json:table:DeviceInfo-no-type-given}


\end{table}

\subsection{EngineConfigTagGroup.json}

\begin{table}[H]
\begin{tabularx}{\textwidth}{l X} 
       \textbf{Name:} & EngineConfig \\ 
	   \textbf{Context:} & com.adlinktech.vision \\ 
	   \textbf{Version:} & v1.0 \\ 
	   \textbf{Description:} & Inference Engine Publishes its current \& desired configuration \\ 
	   \textbf{QoS:} & state \\
	   \textbf{Toplevel Typename:} & \textit{No explicit name given} \\ 
\end{tabularx}
\caption{EngineConfig:com.adlinktech.vision:v1.0}\label{EngineConfigTagGroup.json:table:EngineConfig}
\bigskip
\begin{tabularx}{\textwidth}{l l l l X} 
	 \textbf{Name} & \textbf{Unit} & \textbf{Type} & \textbf{Subtype} & \textbf{Description} \\
	 \midrule
   width & pixels & UINT32 &  & Expected input image width \\
   height & pixels & UINT32 &  & Expected input image height \\
   channels & n/a & UINT32 &  & Expected input image channel count \\
   format & PixelFormat & STRING &  & Expected input image pixel format. See com::vision::models::PixelFormat \\
   crop & N/A & BOOLEAN &  & Declares whether the input image should be cropped to width \& height \\
   resize & N/A & BOOLEAN &  & Declares whether the input image should be resized to width \& height \\
   roi\_x1 & pixels & UINT32 &  & Declares Top-Left X position of Region of Interest if image should be croppped \\
   roi\_y1 & pixels & UINT32 &  & Declares Top-Left Y position of Region of Interest if image should be croppped \\
   framerate & fps & FLOAT32 &  & Specifies the configured/expected framerate for input images treams \\
\end{tabularx}
\caption{EngineConfig:com.adlinktech.vision:v1.0: ImageConfig}\label{EngineConfigTagGroup.json:table:EngineConfig-ImageConfig}

\bigskip
\begin{tabularx}{\textwidth}{l l l l X} 
	 \textbf{Name} & \textbf{Unit} & \textbf{Type} & \textbf{Subtype} & \textbf{Description} \\
	 \midrule
   model\_type & ModelType & STRING &  & Loaded model type see com::vision::models::ModelConfig \\
   model\_name & n/a & STRING &  & Human readable name of the loaded model \\
   mode & PrecisionMode & STRING &  & Loaded model precision. See com::vision::models::PrecisionMode \\
   threshold & Percentage & FLOAT32 &  & Configured confidence/probability threshold \\
   nms\_threshold & Percentage & FLOAT32 &  & Configured Non-Maximum Supression Threshold \\
   batch\_size & Number & UINT32 &  & Configured number of samples sent through the network \\
   output\_type & ModelOutputType & STRING &  & Configured inference output types see com::vision::models::ModelOutputType \\
\end{tabularx}
\caption{EngineConfig:com.adlinktech.vision:v1.0: ModelConfig}\label{EngineConfigTagGroup.json:table:EngineConfig-ModelConfig}

\bigskip
\begin{tabularx}{\textwidth}{l l l l X} 
	 \textbf{Name} & \textbf{Unit} & \textbf{Type} & \textbf{Subtype} & \textbf{Description} \\
	 \midrule
   engine\_id & UUID & STRING &  & Publishing inference engine identifer \\
   name & n/a & STRING &  & Proper name of the inference engine \\
   source & n/a & STRING &  & Producer of the inference engine \\
   version & n/a & STRING &  & Inference engine revision \\
   kind & com.vision.models.inference.EngineKind & STRING &  & Inference engine kind \\
   image & n/a & NVP\_SEQ & ImageConfig & Configured input image expectations \\
   model & n/a & NVP\_SEQ & ModelConfig & Configured (loaded) model metadata \\
\end{tabularx}
\caption{EngineConfig:com.adlinktech.vision:v1.0: \textit{Toplevel Type (no explicit name given)}}\label{EngineConfigTagGroup.json:table:EngineConfig-no-type-given}


\end{table}

\subsection{EngineInfoTagGroup.json}

\begin{table}[H]
\begin{tabularx}{\textwidth}{l X} 
       \textbf{Name:} & EngineInfo \\ 
	   \textbf{Context:} & com.adlinktech.vision \\ 
	   \textbf{Version:} & v1.0 \\ 
	   \textbf{Description:} & Inference Engine State \& Info \\ 
	   \textbf{QoS:} & state \\
	   \textbf{Toplevel Typename:} & \textit{No explicit name given} \\ 
\end{tabularx}
\caption{EngineInfo:com.adlinktech.vision:v1.0}\label{EngineInfoTagGroup.json:table:EngineInfo}
\bigskip
\begin{tabularx}{\textwidth}{l l l l X} 
	 \textbf{Name} & \textbf{Unit} & \textbf{Type} & \textbf{Subtype} & \textbf{Description} \\
	 \midrule
   engine\_id & UUID & STRING &  & Inference engine identifier \\
   state & EngineState & STRING &  & UNKNOWN | ONLINE | WORKING | OFFLINE \\
\end{tabularx}
\caption{EngineInfo:com.adlinktech.vision:v1.0: \textit{Toplevel Type (no explicit name given)}}\label{EngineInfoTagGroup.json:table:EngineInfo-no-type-given}


\end{table}

\subsection{ModelReceiverStatusTagGroup.json}

\begin{table}[H]
\begin{tabularx}{\textwidth}{l X} 
       \textbf{Name:} & ModelReceiverStatus \\ 
	   \textbf{Context:} & com.adlinktech.vision \\ 
	   \textbf{Version:} & v1.0 \\ 
	   \textbf{Description:} & Broadcasts the state of the Inference Engine Model Receiver config status \\ 
	   \textbf{QoS:} & state \\
	   \textbf{Toplevel Typename:} & \textit{No explicit name given} \\ 
\end{tabularx}
\caption{ModelReceiverStatus:com.adlinktech.vision:v1.0}\label{ModelReceiverStatusTagGroup.json:table:ModelReceiverStatus}
\bigskip
\begin{tabularx}{\textwidth}{l l l l X} 
	 \textbf{Name} & \textbf{Unit} & \textbf{Type} & \textbf{Subtype} & \textbf{Description} \\
	 \midrule
   engine\_id & UUID & STRING &  & Source engine identifier \\
   use\_data\_river & N/A & BOOLEAN &  & State of the BinaryArtifact data interface - Is it available? \\
   use\_rest & N/A & BOOLEAN &  & State of the REST Server - Is it available? \\
   rest\_hostname & N/A & STRING &  & Inference engine model receiver REST server hostname or IP Address \\
   rest\_port & N/A & STRING &  & Inference engine model receiver REST server port \\
   rest\_path & N/A & STRING &  & Inference engine model receiver REST server endpoint \\
   rest\_user & N/A & STRING &  & Inference engine model receiver REST server username \\
   rest\_password & N/A & STRING &  & Inference engine model receiver REST server password \\
\end{tabularx}
\caption{ModelReceiverStatus:com.adlinktech.vision:v1.0: \textit{Toplevel Type (no explicit name given)}}\label{ModelReceiverStatusTagGroup.json:table:ModelReceiverStatus-no-type-given}


\end{table}

\subsection{OutputTensorTagGroup.json}

\begin{table}[H]
\begin{tabularx}{\textwidth}{l X} 
       \textbf{Name:} & OutputTensor \\ 
	   \textbf{Context:} & com.adlinktech.vision \\ 
	   \textbf{Version:} & v1.0 \\ 
	   \textbf{Description:} & Generic inference engine results for any model \\ 
	   \textbf{QoS:} & telemetry \\
	   \textbf{Toplevel Typename:} & \textit{No explicit name given} \\ 
\end{tabularx}
\caption{OutputTensor:com.adlinktech.vision:v1.0}\label{OutputTensorTagGroup.json:table:OutputTensor}
\bigskip
\begin{tabularx}{\textwidth}{l l l l X} 
	 \textbf{Name} & \textbf{Unit} & \textbf{Type} & \textbf{Subtype} & \textbf{Description} \\
	 \midrule
   index & length & UINT32 &  & Output tensor index \\
   output & length & UINT32 &  & Named output tensor \\
   rank & length & UINT32\_SEQ &  & Size of output tensor dimension data \\
   data & raw & FLOAT32\_SEQ &  & Data for specified dimension \\
   data\_type & com.vision.models.PrecisionMode & STRING &  & Precision of data sequence \\
\end{tabularx}
\caption{OutputTensor:com.adlinktech.vision:v1.0: TensorData}\label{OutputTensorTagGroup.json:table:OutputTensor-TensorData}

\bigskip
\begin{tabularx}{\textwidth}{l l l l X} 
	 \textbf{Name} & \textbf{Unit} & \textbf{Type} & \textbf{Subtype} & \textbf{Description} \\
	 \midrule
   stream\_id & UUID & STRING &  & ID of the stream fed into the inference engine \\
   engine\_id & UUID & STRING &  & Inference engine identifier \\
   frame\_id & NUM & UINT32 &  & ID of the input video frame fed to the inference engine \\
   dimensions & index & UINT32\_SEQ &  & Total number of dimensions in output tensor \\
   data & n/a & NVP\_SEQ & TensorData & List of OutputTensor Data (the results) \\
\end{tabularx}
\caption{OutputTensor:com.adlinktech.vision:v1.0: \textit{Toplevel Type (no explicit name given)}}\label{OutputTensorTagGroup.json:table:OutputTensor-no-type-given}


\end{table}

\subsection{PerformanceTagGroup.json}

\begin{table}[H]
\begin{tabularx}{\textwidth}{l X} 
       \textbf{Name:} & Performance \\ 
	   \textbf{Context:} & com.adlinktech.vision \\ 
	   \textbf{Version:} & v1.0 \\ 
	   \textbf{Description:} & Inference Engine processing performance update \\ 
	   \textbf{QoS:} & bestEffortTelemetry \\
	   \textbf{Toplevel Typename:} & \textit{No explicit name given} \\ 
\end{tabularx}
\caption{Performance:com.adlinktech.vision:v1.0}\label{PerformanceTagGroup.json:table:Performance}
\bigskip
\begin{tabularx}{\textwidth}{l l l l X} 
	 \textbf{Name} & \textbf{Unit} & \textbf{Type} & \textbf{Subtype} & \textbf{Description} \\
	 \midrule
   engine\_id & UUID & STRING &  & Inference engine identifier \\
   stream\_id & UUID & STRING &  & Stream publisher ID \\
   frame\_id & NUM & UINT32 &  & Frame sample ID \\
   delay & milliseconds & UINT32 &  & Inference processing time \\
\end{tabularx}
\caption{Performance:com.adlinktech.vision:v1.0: \textit{Toplevel Type (no explicit name given)}}\label{PerformanceTagGroup.json:table:Performance-no-type-given}


\end{table}

\subsection{SegmentationTagGroup.json}

\begin{table}[H]
\begin{tabularx}{\textwidth}{l X} 
       \textbf{Name:} & Segmentation \\ 
	   \textbf{Context:} & com.adlinktech.vision \\ 
	   \textbf{Version:} & v1.0 \\ 
	   \textbf{Description:} & Inference engine results for classification model \\ 
	   \textbf{QoS:} & telemetry \\
	   \textbf{Toplevel Typename:} & \textit{No explicit name given} \\ 
\end{tabularx}
\caption{Segmentation:com.adlinktech.vision:v1.0}\label{SegmentationTagGroup.json:table:Segmentation}
\bigskip
\begin{tabularx}{\textwidth}{l l l l X} 
	 \textbf{Name} & \textbf{Unit} & \textbf{Type} & \textbf{Subtype} & \textbf{Description} \\
	 \midrule
   id & UUID & INT32 &  & Segmentation mask index/id \\
   width & pixels & INT32 &  & Segmentation mask frame width \\
   height & pixels & INT32 &  & Segmentation mask frame height \\
   size & bytes & INT32 &  & Segmentation mask payload size (bytes) \\
   channel & index & INT32 &  & Segmentation mask output channel \\
   color & Color & STRING &  & Segmentation mask color (\#XXXXXX) \\
   label & UUID & STRING &  & Classification label name \\
   probability & Percentage & FLOAT32 &  & Network confidence \\
   data & bytes & BYTE\_SEQ &  & Segmentation mask frame data \\
\end{tabularx}
\caption{Segmentation:com.adlinktech.vision:v1.0: SegmentationData}\label{SegmentationTagGroup.json:table:Segmentation-SegmentationData}

\bigskip
\begin{tabularx}{\textwidth}{l l l l X} 
	 \textbf{Name} & \textbf{Unit} & \textbf{Type} & \textbf{Subtype} & \textbf{Description} \\
	 \midrule
   engine\_id & UUID & STRING &  & Inference engine identifier \\
   stream\_id & UUID & STRING &  & ID of the stream fed into the inference engine \\
   frame\_id & NUM & UINT32 &  & ID of the input video frame fed to the inference engine \\
   inference\_id & NUM & UINT32 &  & Segmentation mask instance in the result array \\
   data & n/a & NVP\_SEQ & SegmentationData & List of Segmentation Data (the results) \\
\end{tabularx}
\caption{Segmentation:com.adlinktech.vision:v1.0: \textit{Toplevel Type (no explicit name given)}}\label{SegmentationTagGroup.json:table:Segmentation-no-type-given}


\end{table}

\subsection{StreamRequestTagGroup.json}

\begin{table}[H]
\begin{tabularx}{\textwidth}{l X} 
       \textbf{Name:} & StreamRequest \\ 
	   \textbf{Context:} & com.adlinktech.vision \\ 
	   \textbf{Version:} & v1.0 \\ 
	   \textbf{Description:} & Request model for initating device video streams \\ 
	   \textbf{QoS:} & event \\
	   \textbf{Toplevel Typename:} & \textit{No explicit name given} \\ 
\end{tabularx}
\caption{StreamRequest:com.adlinktech.vision:v1.0}\label{StreamRequestTagGroup.json:table:StreamRequest}
\bigskip
\begin{tabularx}{\textwidth}{l l l l X} 
	 \textbf{Name} & \textbf{Unit} & \textbf{Type} & \textbf{Subtype} & \textbf{Description} \\
	 \midrule
   stream\_id & UUID & STRING &  & ID of the stream to control \\
   enable\_stream & n/a & BOOLEAN &  & Enable or disable the stream \\
   framerate & frames per second & FLOAT32 &  & Specify the framerate that should be streamed \\
\end{tabularx}
\caption{StreamRequest:com.adlinktech.vision:v1.0: \textit{Toplevel Type (no explicit name given)}}\label{StreamRequestTagGroup.json:table:StreamRequest-no-type-given}


\end{table}

\subsection{TrainingStreamerConfigTagGroup.json}

\begin{table}[H]
\begin{tabularx}{\textwidth}{l X} 
       \textbf{Name:} & TrainingStreamerConfig \\ 
	   \textbf{Context:} & com.adlinktech.vision \\ 
	   \textbf{Version:} & v1.0 \\ 
	   \textbf{Description:} & Reports if a training streamer status is change. \\ 
	   \textbf{QoS:} & state \\
	   \textbf{Toplevel Typename:} & \textit{No explicit name given} \\ 
\end{tabularx}
\caption{TrainingStreamerConfig:com.adlinktech.vision:v1.0}\label{TrainingStreamerConfigTagGroup.json:table:TrainingStreamerConfig}
\bigskip
\begin{tabularx}{\textwidth}{l l l l X} 
	 \textbf{Name} & \textbf{Unit} & \textbf{Type} & \textbf{Subtype} & \textbf{Description} \\
	 \midrule
   StreamId & NA & STRING &  & Stream id \\
   Status & NA & BOOLEAN &  & on (true) or off (false) \\
   NumberOfFrame & NA & INT32 &  & Number of frames per time unit, 0 means no downsampling \\
   TimeUnit & NA & STRING &  & Time unit, could be Second/Minute/Hour \\
\end{tabularx}
\caption{TrainingStreamerConfig:com.adlinktech.vision:v1.0: StreamInfo}\label{TrainingStreamerConfigTagGroup.json:table:TrainingStreamerConfig-StreamInfo}

\bigskip
\begin{tabularx}{\textwidth}{l l l l X} 
	 \textbf{Name} & \textbf{Unit} & \textbf{Type} & \textbf{Subtype} & \textbf{Description} \\
	 \midrule
   FrameStorage & NA & STRING &  & Save location in local or remote. \\
   StreamId & NA & NVP\_SEQ & StreamInfo & List of stream id \\
\end{tabularx}
\caption{TrainingStreamerConfig:com.adlinktech.vision:v1.0: \textit{Toplevel Type (no explicit name given)}}\label{TrainingStreamerConfigTagGroup.json:table:TrainingStreamerConfig-no-type-given}


\end{table}

\subsection{VideoFrameTagGroup.json}

\begin{table}[H]
\begin{tabularx}{\textwidth}{l X} 
       \textbf{Name:} & VideoFrame \\ 
	   \textbf{Context:} & com.adlinktech.vision \\ 
	   \textbf{Version:} & v1.0 \\ 
	   \textbf{Description:} & Video frame sample \\ 
	   \textbf{QoS:} & video \\
	   \textbf{Toplevel Typename:} & \textit{No explicit name given} \\ 
\end{tabularx}
\caption{VideoFrame:com.adlinktech.vision:v1.0}\label{VideoFrameTagGroup.json:table:VideoFrame}
\bigskip
\begin{tabularx}{\textwidth}{l l l l X} 
	 \textbf{Name} & \textbf{Unit} & \textbf{Type} & \textbf{Subtype} & \textbf{Description} \\
	 \midrule
   stream\_id & UUID & STRING &  & Stream publisher ID \\
   frame\_id & NUM & UINT32 &  & Frame sample ID \\
   timestamp & time & INT64 &  & Time of image capture event \\
   data & Frame data & BYTE\_SEQ &  & Video frame data \\
   width & Pixels & UINT32 &  & Frame width \\
   height & Pixels & UINT32 &  & Frame height \\
   channels & Number of channels & UINT32 &  & Channels \\
   size & Size & UINT32 &  & Data size \\
   format & PixelFormat & STRING &  & Pixel format using OpenCV Definitions \\
   compression & CompressionKind & STRING &  & Compression technology used for video frame \\
   framerate & fps & FLOAT32 &  & Frame transmission frequency \\
\end{tabularx}
\caption{VideoFrame:com.adlinktech.vision:v1.0: \textit{Toplevel Type (no explicit name given)}}\label{VideoFrameTagGroup.json:table:VideoFrame-no-type-given}


\end{table}

\section{TagGroup Definition Cross-Reference}
     \begin{longtable}[Hl]{l l} 
	 \caption{TagGroup Definition Cross-Reference}\label{tab:TagGroupDefinitionCrossReference}\\
	 \textbf{TagGroup} & \textbf{Filename} \\
	 \midrule\endhead
AITagGroup:com.adlinktech.ai:v1.0 & AITagGroup.json \\
Acknowledge:com.adlinktech.vision:v1.0 & AcknowledgeTagGroup.json \\
Acknowledge:com.adlinktech.vision:v1.0 & AlarmTagGroup.json \\
Acknowledge:com.adlinktech.vision:v1.0 & BinaryArtifactTagGroup.json \\
Acknowledge:com.adlinktech.vision:v1.0 & CameraConnectRequestTagGroup.json \\
Acknowledge:com.adlinktech.vision:v1.0 & CaptureStartTagGroup.json \\
Acknowledge:com.adlinktech.vision:v1.0 & CaptureStopTagGroup.json \\
Acknowledge:com.adlinktech.vision:v1.0 & ClassificationTagGroup.json \\
Acknowledge:com.adlinktech.vision:v1.0 & ConfigActualTagGroup.json \\
Acknowledge:com.adlinktech.vision:v1.0 & ConfigRequestTagGroup.json \\
Acknowledge:com.adlinktech.vision:v1.0 & DetectionBoxTagGroup.json \\
Acknowledge:com.adlinktech.vision:v1.0 & DetectionPointTagGroup.json \\
Acknowledge:com.adlinktech.vision:v1.0 & DeviceErrorTagGroup.json \\
Acknowledge:com.adlinktech.vision:v1.0 & DeviceInfoTagGroup.json \\
Acknowledge:com.adlinktech.vision:v1.0 & EngineConfigTagGroup.json \\
Acknowledge:com.adlinktech.vision:v1.0 & EngineInfoTagGroup.json \\
Acknowledge:com.adlinktech.vision:v1.0 & ModelReceiverStatusTagGroup.json \\
Acknowledge:com.adlinktech.vision:v1.0 & OutputTensorTagGroup.json \\
Acknowledge:com.adlinktech.vision:v1.0 & PerformanceTagGroup.json \\
Acknowledge:com.adlinktech.vision:v1.0 & SegmentationTagGroup.json \\
Acknowledge:com.adlinktech.vision:v1.0 & StreamRequestTagGroup.json \\
Acknowledge:com.adlinktech.vision:v1.0 & TrainingStreamerConfigTagGroup.json \\
Acknowledge:com.adlinktech.vision:v1.0 & VideoFrameTagGroup.json \\
Alarm:com.adlinktech.vision:v1.0 & AlarmTagGroup.json \\
Alarm:com.adlinktech.vision:v1.0 & BinaryArtifactTagGroup.json \\
Alarm:com.adlinktech.vision:v1.0 & CameraConnectRequestTagGroup.json \\
Alarm:com.adlinktech.vision:v1.0 & CaptureStartTagGroup.json \\
Alarm:com.adlinktech.vision:v1.0 & CaptureStopTagGroup.json \\
Alarm:com.adlinktech.vision:v1.0 & ClassificationTagGroup.json \\
Alarm:com.adlinktech.vision:v1.0 & ConfigActualTagGroup.json \\
Alarm:com.adlinktech.vision:v1.0 & ConfigRequestTagGroup.json \\
Alarm:com.adlinktech.vision:v1.0 & DetectionBoxTagGroup.json \\
Alarm:com.adlinktech.vision:v1.0 & DetectionPointTagGroup.json \\
Alarm:com.adlinktech.vision:v1.0 & DeviceErrorTagGroup.json \\
Alarm:com.adlinktech.vision:v1.0 & DeviceInfoTagGroup.json \\
Alarm:com.adlinktech.vision:v1.0 & EngineConfigTagGroup.json \\
Alarm:com.adlinktech.vision:v1.0 & EngineInfoTagGroup.json \\
Alarm:com.adlinktech.vision:v1.0 & ModelReceiverStatusTagGroup.json \\
Alarm:com.adlinktech.vision:v1.0 & OutputTensorTagGroup.json \\
Alarm:com.adlinktech.vision:v1.0 & PerformanceTagGroup.json \\
Alarm:com.adlinktech.vision:v1.0 & SegmentationTagGroup.json \\
Alarm:com.adlinktech.vision:v1.0 & StreamRequestTagGroup.json \\
Alarm:com.adlinktech.vision:v1.0 & TrainingStreamerConfigTagGroup.json \\
Alarm:com.adlinktech.vision:v1.0 & VideoFrameTagGroup.json \\
BinaryArtifact:com.adlinktech.vision:v1.0 & BinaryArtifactTagGroup.json \\
BinaryArtifact:com.adlinktech.vision:v1.0 & CameraConnectRequestTagGroup.json \\
BinaryArtifact:com.adlinktech.vision:v1.0 & CaptureStartTagGroup.json \\
BinaryArtifact:com.adlinktech.vision:v1.0 & CaptureStopTagGroup.json \\
BinaryArtifact:com.adlinktech.vision:v1.0 & ClassificationTagGroup.json \\
BinaryArtifact:com.adlinktech.vision:v1.0 & ConfigActualTagGroup.json \\
BinaryArtifact:com.adlinktech.vision:v1.0 & ConfigRequestTagGroup.json \\
BinaryArtifact:com.adlinktech.vision:v1.0 & DetectionBoxTagGroup.json \\
BinaryArtifact:com.adlinktech.vision:v1.0 & DetectionPointTagGroup.json \\
BinaryArtifact:com.adlinktech.vision:v1.0 & DeviceErrorTagGroup.json \\
BinaryArtifact:com.adlinktech.vision:v1.0 & DeviceInfoTagGroup.json \\
BinaryArtifact:com.adlinktech.vision:v1.0 & EngineConfigTagGroup.json \\
BinaryArtifact:com.adlinktech.vision:v1.0 & EngineInfoTagGroup.json \\
BinaryArtifact:com.adlinktech.vision:v1.0 & ModelReceiverStatusTagGroup.json \\
BinaryArtifact:com.adlinktech.vision:v1.0 & OutputTensorTagGroup.json \\
BinaryArtifact:com.adlinktech.vision:v1.0 & PerformanceTagGroup.json \\
BinaryArtifact:com.adlinktech.vision:v1.0 & SegmentationTagGroup.json \\
BinaryArtifact:com.adlinktech.vision:v1.0 & StreamRequestTagGroup.json \\
BinaryArtifact:com.adlinktech.vision:v1.0 & TrainingStreamerConfigTagGroup.json \\
BinaryArtifact:com.adlinktech.vision:v1.0 & VideoFrameTagGroup.json \\
CameraConnectRequest:com.adlinktech.vision:v1.0 & CameraConnectRequestTagGroup.json \\
CameraConnectRequest:com.adlinktech.vision:v1.0 & CaptureStartTagGroup.json \\
CameraConnectRequest:com.adlinktech.vision:v1.0 & CaptureStopTagGroup.json \\
CameraConnectRequest:com.adlinktech.vision:v1.0 & ClassificationTagGroup.json \\
CameraConnectRequest:com.adlinktech.vision:v1.0 & ConfigActualTagGroup.json \\
CameraConnectRequest:com.adlinktech.vision:v1.0 & ConfigRequestTagGroup.json \\
CameraConnectRequest:com.adlinktech.vision:v1.0 & DetectionBoxTagGroup.json \\
CameraConnectRequest:com.adlinktech.vision:v1.0 & DetectionPointTagGroup.json \\
CameraConnectRequest:com.adlinktech.vision:v1.0 & DeviceErrorTagGroup.json \\
CameraConnectRequest:com.adlinktech.vision:v1.0 & DeviceInfoTagGroup.json \\
CameraConnectRequest:com.adlinktech.vision:v1.0 & EngineConfigTagGroup.json \\
CameraConnectRequest:com.adlinktech.vision:v1.0 & EngineInfoTagGroup.json \\
CameraConnectRequest:com.adlinktech.vision:v1.0 & ModelReceiverStatusTagGroup.json \\
CameraConnectRequest:com.adlinktech.vision:v1.0 & OutputTensorTagGroup.json \\
CameraConnectRequest:com.adlinktech.vision:v1.0 & PerformanceTagGroup.json \\
CameraConnectRequest:com.adlinktech.vision:v1.0 & SegmentationTagGroup.json \\
CameraConnectRequest:com.adlinktech.vision:v1.0 & StreamRequestTagGroup.json \\
CameraConnectRequest:com.adlinktech.vision:v1.0 & TrainingStreamerConfigTagGroup.json \\
CameraConnectRequest:com.adlinktech.vision:v1.0 & VideoFrameTagGroup.json \\
CaptureStart:com.adlinktech.vision:v1.0 & CaptureStartTagGroup.json \\
CaptureStart:com.adlinktech.vision:v1.0 & CaptureStopTagGroup.json \\
CaptureStart:com.adlinktech.vision:v1.0 & ClassificationTagGroup.json \\
CaptureStart:com.adlinktech.vision:v1.0 & ConfigActualTagGroup.json \\
CaptureStart:com.adlinktech.vision:v1.0 & ConfigRequestTagGroup.json \\
CaptureStart:com.adlinktech.vision:v1.0 & DetectionBoxTagGroup.json \\
CaptureStart:com.adlinktech.vision:v1.0 & DetectionPointTagGroup.json \\
CaptureStart:com.adlinktech.vision:v1.0 & DeviceErrorTagGroup.json \\
CaptureStart:com.adlinktech.vision:v1.0 & DeviceInfoTagGroup.json \\
CaptureStart:com.adlinktech.vision:v1.0 & EngineConfigTagGroup.json \\
CaptureStart:com.adlinktech.vision:v1.0 & EngineInfoTagGroup.json \\
CaptureStart:com.adlinktech.vision:v1.0 & ModelReceiverStatusTagGroup.json \\
CaptureStart:com.adlinktech.vision:v1.0 & OutputTensorTagGroup.json \\
CaptureStart:com.adlinktech.vision:v1.0 & PerformanceTagGroup.json \\
CaptureStart:com.adlinktech.vision:v1.0 & SegmentationTagGroup.json \\
CaptureStart:com.adlinktech.vision:v1.0 & StreamRequestTagGroup.json \\
CaptureStart:com.adlinktech.vision:v1.0 & TrainingStreamerConfigTagGroup.json \\
CaptureStart:com.adlinktech.vision:v1.0 & VideoFrameTagGroup.json \\
CaptureStop:com.adlinktech.vision:v1.0 & CaptureStopTagGroup.json \\
CaptureStop:com.adlinktech.vision:v1.0 & ClassificationTagGroup.json \\
CaptureStop:com.adlinktech.vision:v1.0 & ConfigActualTagGroup.json \\
CaptureStop:com.adlinktech.vision:v1.0 & ConfigRequestTagGroup.json \\
CaptureStop:com.adlinktech.vision:v1.0 & DetectionBoxTagGroup.json \\
CaptureStop:com.adlinktech.vision:v1.0 & DetectionPointTagGroup.json \\
CaptureStop:com.adlinktech.vision:v1.0 & DeviceErrorTagGroup.json \\
CaptureStop:com.adlinktech.vision:v1.0 & DeviceInfoTagGroup.json \\
CaptureStop:com.adlinktech.vision:v1.0 & EngineConfigTagGroup.json \\
CaptureStop:com.adlinktech.vision:v1.0 & EngineInfoTagGroup.json \\
CaptureStop:com.adlinktech.vision:v1.0 & ModelReceiverStatusTagGroup.json \\
CaptureStop:com.adlinktech.vision:v1.0 & OutputTensorTagGroup.json \\
CaptureStop:com.adlinktech.vision:v1.0 & PerformanceTagGroup.json \\
CaptureStop:com.adlinktech.vision:v1.0 & SegmentationTagGroup.json \\
CaptureStop:com.adlinktech.vision:v1.0 & StreamRequestTagGroup.json \\
CaptureStop:com.adlinktech.vision:v1.0 & TrainingStreamerConfigTagGroup.json \\
CaptureStop:com.adlinktech.vision:v1.0 & VideoFrameTagGroup.json \\
Classification:com.adlinktech.vision:v1.0 & ClassificationTagGroup.json \\
Classification:com.adlinktech.vision:v1.0 & ConfigActualTagGroup.json \\
Classification:com.adlinktech.vision:v1.0 & ConfigRequestTagGroup.json \\
Classification:com.adlinktech.vision:v1.0 & DetectionBoxTagGroup.json \\
Classification:com.adlinktech.vision:v1.0 & DetectionPointTagGroup.json \\
Classification:com.adlinktech.vision:v1.0 & DeviceErrorTagGroup.json \\
Classification:com.adlinktech.vision:v1.0 & DeviceInfoTagGroup.json \\
Classification:com.adlinktech.vision:v1.0 & EngineConfigTagGroup.json \\
Classification:com.adlinktech.vision:v1.0 & EngineInfoTagGroup.json \\
Classification:com.adlinktech.vision:v1.0 & ModelReceiverStatusTagGroup.json \\
Classification:com.adlinktech.vision:v1.0 & OutputTensorTagGroup.json \\
Classification:com.adlinktech.vision:v1.0 & PerformanceTagGroup.json \\
Classification:com.adlinktech.vision:v1.0 & SegmentationTagGroup.json \\
Classification:com.adlinktech.vision:v1.0 & StreamRequestTagGroup.json \\
Classification:com.adlinktech.vision:v1.0 & TrainingStreamerConfigTagGroup.json \\
Classification:com.adlinktech.vision:v1.0 & VideoFrameTagGroup.json \\
ConfigActual:com.adlinktech.vision:v1.0 & ConfigActualTagGroup.json \\
ConfigActual:com.adlinktech.vision:v1.0 & ConfigRequestTagGroup.json \\
ConfigActual:com.adlinktech.vision:v1.0 & DetectionBoxTagGroup.json \\
ConfigActual:com.adlinktech.vision:v1.0 & DetectionPointTagGroup.json \\
ConfigActual:com.adlinktech.vision:v1.0 & DeviceErrorTagGroup.json \\
ConfigActual:com.adlinktech.vision:v1.0 & DeviceInfoTagGroup.json \\
ConfigActual:com.adlinktech.vision:v1.0 & EngineConfigTagGroup.json \\
ConfigActual:com.adlinktech.vision:v1.0 & EngineInfoTagGroup.json \\
ConfigActual:com.adlinktech.vision:v1.0 & ModelReceiverStatusTagGroup.json \\
ConfigActual:com.adlinktech.vision:v1.0 & OutputTensorTagGroup.json \\
ConfigActual:com.adlinktech.vision:v1.0 & PerformanceTagGroup.json \\
ConfigActual:com.adlinktech.vision:v1.0 & SegmentationTagGroup.json \\
ConfigActual:com.adlinktech.vision:v1.0 & StreamRequestTagGroup.json \\
ConfigActual:com.adlinktech.vision:v1.0 & TrainingStreamerConfigTagGroup.json \\
ConfigActual:com.adlinktech.vision:v1.0 & VideoFrameTagGroup.json \\
ConfigRequest:com.adlinktech.vision:v1.0 & ConfigRequestTagGroup.json \\
ConfigRequest:com.adlinktech.vision:v1.0 & DetectionBoxTagGroup.json \\
ConfigRequest:com.adlinktech.vision:v1.0 & DetectionPointTagGroup.json \\
ConfigRequest:com.adlinktech.vision:v1.0 & DeviceErrorTagGroup.json \\
ConfigRequest:com.adlinktech.vision:v1.0 & DeviceInfoTagGroup.json \\
ConfigRequest:com.adlinktech.vision:v1.0 & EngineConfigTagGroup.json \\
ConfigRequest:com.adlinktech.vision:v1.0 & EngineInfoTagGroup.json \\
ConfigRequest:com.adlinktech.vision:v1.0 & ModelReceiverStatusTagGroup.json \\
ConfigRequest:com.adlinktech.vision:v1.0 & OutputTensorTagGroup.json \\
ConfigRequest:com.adlinktech.vision:v1.0 & PerformanceTagGroup.json \\
ConfigRequest:com.adlinktech.vision:v1.0 & SegmentationTagGroup.json \\
ConfigRequest:com.adlinktech.vision:v1.0 & StreamRequestTagGroup.json \\
ConfigRequest:com.adlinktech.vision:v1.0 & TrainingStreamerConfigTagGroup.json \\
ConfigRequest:com.adlinktech.vision:v1.0 & VideoFrameTagGroup.json \\
CreateStream:com.adlinktech.vision:0.1.0 & CreateStreamTagGroup.json \\
CreateStream:com.adlinktech.vision:0.1.0 & DeleteStreamTagGroup.json \\
CreateStream:com.adlinktech.vision:0.1.0 & StreamViewerConfigTagGroup.json \\
DeleteStream:com.adlinktech.vision:0.1.0 & DeleteStreamTagGroup.json \\
DeleteStream:com.adlinktech.vision:0.1.0 & StreamViewerConfigTagGroup.json \\
DetectionBox:com.adlinktech.vision:v1.0 & DetectionBoxTagGroup.json \\
DetectionBox:com.adlinktech.vision:v1.0 & DetectionPointTagGroup.json \\
DetectionBox:com.adlinktech.vision:v1.0 & DeviceErrorTagGroup.json \\
DetectionBox:com.adlinktech.vision:v1.0 & DeviceInfoTagGroup.json \\
DetectionBox:com.adlinktech.vision:v1.0 & EngineConfigTagGroup.json \\
DetectionBox:com.adlinktech.vision:v1.0 & EngineInfoTagGroup.json \\
DetectionBox:com.adlinktech.vision:v1.0 & ModelReceiverStatusTagGroup.json \\
DetectionBox:com.adlinktech.vision:v1.0 & OutputTensorTagGroup.json \\
DetectionBox:com.adlinktech.vision:v1.0 & PerformanceTagGroup.json \\
DetectionBox:com.adlinktech.vision:v1.0 & SegmentationTagGroup.json \\
DetectionBox:com.adlinktech.vision:v1.0 & StreamRequestTagGroup.json \\
DetectionBox:com.adlinktech.vision:v1.0 & TrainingStreamerConfigTagGroup.json \\
DetectionBox:com.adlinktech.vision:v1.0 & VideoFrameTagGroup.json \\
DetectionPoint:com.adlinktech.vision:v1.0 & DetectionPointTagGroup.json \\
DetectionPoint:com.adlinktech.vision:v1.0 & DeviceErrorTagGroup.json \\
DetectionPoint:com.adlinktech.vision:v1.0 & DeviceInfoTagGroup.json \\
DetectionPoint:com.adlinktech.vision:v1.0 & EngineConfigTagGroup.json \\
DetectionPoint:com.adlinktech.vision:v1.0 & EngineInfoTagGroup.json \\
DetectionPoint:com.adlinktech.vision:v1.0 & ModelReceiverStatusTagGroup.json \\
DetectionPoint:com.adlinktech.vision:v1.0 & OutputTensorTagGroup.json \\
DetectionPoint:com.adlinktech.vision:v1.0 & PerformanceTagGroup.json \\
DetectionPoint:com.adlinktech.vision:v1.0 & SegmentationTagGroup.json \\
DetectionPoint:com.adlinktech.vision:v1.0 & StreamRequestTagGroup.json \\
DetectionPoint:com.adlinktech.vision:v1.0 & TrainingStreamerConfigTagGroup.json \\
DetectionPoint:com.adlinktech.vision:v1.0 & VideoFrameTagGroup.json \\
DeviceError:com.adlinktech.vision:v1.0 & DeviceErrorTagGroup.json \\
DeviceError:com.adlinktech.vision:v1.0 & DeviceInfoTagGroup.json \\
DeviceError:com.adlinktech.vision:v1.0 & EngineConfigTagGroup.json \\
DeviceError:com.adlinktech.vision:v1.0 & EngineInfoTagGroup.json \\
DeviceError:com.adlinktech.vision:v1.0 & ModelReceiverStatusTagGroup.json \\
DeviceError:com.adlinktech.vision:v1.0 & OutputTensorTagGroup.json \\
DeviceError:com.adlinktech.vision:v1.0 & PerformanceTagGroup.json \\
DeviceError:com.adlinktech.vision:v1.0 & SegmentationTagGroup.json \\
DeviceError:com.adlinktech.vision:v1.0 & StreamRequestTagGroup.json \\
DeviceError:com.adlinktech.vision:v1.0 & TrainingStreamerConfigTagGroup.json \\
DeviceError:com.adlinktech.vision:v1.0 & VideoFrameTagGroup.json \\
DeviceInfo:com.adlinktech.vision:v1.0 & DeviceInfoTagGroup.json \\
DeviceInfo:com.adlinktech.vision:v1.0 & EngineConfigTagGroup.json \\
DeviceInfo:com.adlinktech.vision:v1.0 & EngineInfoTagGroup.json \\
DeviceInfo:com.adlinktech.vision:v1.0 & ModelReceiverStatusTagGroup.json \\
DeviceInfo:com.adlinktech.vision:v1.0 & OutputTensorTagGroup.json \\
DeviceInfo:com.adlinktech.vision:v1.0 & PerformanceTagGroup.json \\
DeviceInfo:com.adlinktech.vision:v1.0 & SegmentationTagGroup.json \\
DeviceInfo:com.adlinktech.vision:v1.0 & StreamRequestTagGroup.json \\
DeviceInfo:com.adlinktech.vision:v1.0 & TrainingStreamerConfigTagGroup.json \\
DeviceInfo:com.adlinktech.vision:v1.0 & VideoFrameTagGroup.json \\
EngineConfig:com.adlinktech.vision:v1.0 & EngineConfigTagGroup.json \\
EngineConfig:com.adlinktech.vision:v1.0 & EngineInfoTagGroup.json \\
EngineConfig:com.adlinktech.vision:v1.0 & ModelReceiverStatusTagGroup.json \\
EngineConfig:com.adlinktech.vision:v1.0 & OutputTensorTagGroup.json \\
EngineConfig:com.adlinktech.vision:v1.0 & PerformanceTagGroup.json \\
EngineConfig:com.adlinktech.vision:v1.0 & SegmentationTagGroup.json \\
EngineConfig:com.adlinktech.vision:v1.0 & StreamRequestTagGroup.json \\
EngineConfig:com.adlinktech.vision:v1.0 & TrainingStreamerConfigTagGroup.json \\
EngineConfig:com.adlinktech.vision:v1.0 & VideoFrameTagGroup.json \\
EngineInfo:com.adlinktech.vision:v1.0 & EngineInfoTagGroup.json \\
EngineInfo:com.adlinktech.vision:v1.0 & ModelReceiverStatusTagGroup.json \\
EngineInfo:com.adlinktech.vision:v1.0 & OutputTensorTagGroup.json \\
EngineInfo:com.adlinktech.vision:v1.0 & PerformanceTagGroup.json \\
EngineInfo:com.adlinktech.vision:v1.0 & SegmentationTagGroup.json \\
EngineInfo:com.adlinktech.vision:v1.0 & StreamRequestTagGroup.json \\
EngineInfo:com.adlinktech.vision:v1.0 & TrainingStreamerConfigTagGroup.json \\
EngineInfo:com.adlinktech.vision:v1.0 & VideoFrameTagGroup.json \\
ModelReceiverStatus:com.adlinktech.vision:v1.0 & ModelReceiverStatusTagGroup.json \\
ModelReceiverStatus:com.adlinktech.vision:v1.0 & OutputTensorTagGroup.json \\
ModelReceiverStatus:com.adlinktech.vision:v1.0 & PerformanceTagGroup.json \\
ModelReceiverStatus:com.adlinktech.vision:v1.0 & SegmentationTagGroup.json \\
ModelReceiverStatus:com.adlinktech.vision:v1.0 & StreamRequestTagGroup.json \\
ModelReceiverStatus:com.adlinktech.vision:v1.0 & TrainingStreamerConfigTagGroup.json \\
ModelReceiverStatus:com.adlinktech.vision:v1.0 & VideoFrameTagGroup.json \\
OutputTensor:com.adlinktech.vision:v1.0 & OutputTensorTagGroup.json \\
OutputTensor:com.adlinktech.vision:v1.0 & PerformanceTagGroup.json \\
OutputTensor:com.adlinktech.vision:v1.0 & SegmentationTagGroup.json \\
OutputTensor:com.adlinktech.vision:v1.0 & StreamRequestTagGroup.json \\
OutputTensor:com.adlinktech.vision:v1.0 & TrainingStreamerConfigTagGroup.json \\
OutputTensor:com.adlinktech.vision:v1.0 & VideoFrameTagGroup.json \\
Performance:com.adlinktech.vision:v1.0 & PerformanceTagGroup.json \\
Performance:com.adlinktech.vision:v1.0 & SegmentationTagGroup.json \\
Performance:com.adlinktech.vision:v1.0 & StreamRequestTagGroup.json \\
Performance:com.adlinktech.vision:v1.0 & TrainingStreamerConfigTagGroup.json \\
Performance:com.adlinktech.vision:v1.0 & VideoFrameTagGroup.json \\
Segmentation:com.adlinktech.vision:v1.0 & SegmentationTagGroup.json \\
Segmentation:com.adlinktech.vision:v1.0 & StreamRequestTagGroup.json \\
Segmentation:com.adlinktech.vision:v1.0 & TrainingStreamerConfigTagGroup.json \\
Segmentation:com.adlinktech.vision:v1.0 & VideoFrameTagGroup.json \\
StreamRequest:com.adlinktech.vision:v1.0 & StreamRequestTagGroup.json \\
StreamRequest:com.adlinktech.vision:v1.0 & TrainingStreamerConfigTagGroup.json \\
StreamRequest:com.adlinktech.vision:v1.0 & VideoFrameTagGroup.json \\
StreamViewerConfig:com.adlinktech.vision:0.1.0 & StreamViewerConfigTagGroup.json \\
TrainingStreamerConfig:com.adlinktech.vision:v1.0 & TrainingStreamerConfigTagGroup.json \\
TrainingStreamerConfig:com.adlinktech.vision:v1.0 & VideoFrameTagGroup.json \\
VideoFrame:com.adlinktech.vision:v1.0 & VideoFrameTagGroup.json \\
configuration:com.adlinktech.usb2405:v1.0 & DAQStateTagGroup.json \\
meta_data:com.adlinktech.usb2405:v1.0 & ChannelMetaDataTagGroup.json \\
meta_data:com.adlinktech.usb2405:v1.0 & DAQStateTagGroup.json \\
raw_data:com.adlinktech.usb2405:v1.0 & ChannelDataTagGroup.json \\
raw_data:com.adlinktech.usb2405:v1.0 & ChannelMetaDataTagGroup.json \\
raw_data:com.adlinktech.usb2405:v1.0 & DAQStateTagGroup.json \\
  \end{longtable}
%\end{landscape}
\end{document}
